\chapter{Исследовательская часть}

\section{Результаты решения задачи}

В таблице~\ref{tab:mixed-numbering} представлены результаты проверки смешения типов нумерации иллюстрирующих элементов для приложенного набора файлов.

\begin{table}[ht]
	\centering
	\small
	\setlength{\tabcolsep}{4pt}
	\renewcommand{\arraystretch}{1.15}
	
	\caption{Результаты проверки смешения типов нумерации иллюстрирующих элементов}
	\label{tab:mixed-numbering}
	
	\begin{tabular}{|p{4.2cm}|p{5.3cm}|p{5cm}|}
		\hline
		\textbf{Название pdf-файла} &
		\textbf{Признак успешного нахождения подстроки} &
		\textbf{Координаты первого нахождения подстроки} \\
		\hline
		\texttt{\detokenize{_00.pdf}}      & ложь   & -- \\
		\hline
		\texttt{\detokenize{_05.pdf}}      & ложь   & -- \\
		\hline
		\texttt{\detokenize{_06.pdf}}      & ложь   & -- \\
		\hline
		\texttt{\detokenize{_07.pdf}}      & ложь   & -- \\
		\hline
		\texttt{\detokenize{_08.pdf}}      & ложь   & -- \\
		\hline
		\texttt{\detokenize{_09.pdf}}      & ложь   & -- \\
		\hline
		\texttt{\detokenize{_10.pdf}}      & ложь   & -- \\
		\hline
		\texttt{\detokenize{_65-1.pdf}}    & ложь   & -- \\
		\hline
		\texttt{main (2).pdf}              & истина & стр.~24, строка~9 \\
		\hline
		\texttt{ВКР Селез.pdf}             & ложь   & -- \\
		\hline
	\end{tabular}
\end{table}



\section{Оценка сложности алгоритма}

Пусть:
\begin{enumerate}
	\item $P$ --- число страниц в файле;
	\item $L_{i}$ --- число строк на странице $i$;
	\item $C_{i}$ --- число символов на странице $i$.
\end{enumerate}

\bigskip

На странице $i$ алгоритм получает текст длиной  $C_{i}$ и делит его на строки, т.~е. осуществляет один проход по всем символам страницы: $O(C_{i})$.

Дальше для каждой строки $j$ длиной $|line_{i, j}|$ происходит поиск по регулярному выражению поиск работает линейно по длине строки: $O(|line_{i, j}|)$.

Тогда сложность поиска на странице по формуле~\ref{eq1}:


\begin{equation}
	\label{eq1}
	\begin{gathered}
	O(C_{i}) + \sum_{j=1}^{L_{i}} O(|line_{i, j}|).
	\end{gathered}
\end{equation}

Но сумма длин всех строк на странице --- это и есть $C_{i}$, то есть по формуле~\ref{eq2}:

\begin{equation}
	\label{eq2}
	\begin{gathered}
		\sum_{j=1}^{L_{i}} O(|line_{i, j}|) \approx C_{i}.
	\end{gathered}
\end{equation}

Тогда сложность поиска по всему документу~\ref{eq3}:

\begin{equation}
	\label{eq3}
	\begin{gathered}
		O(\sum_{i=1}^{P} C_i) \approx O(N),
	\end{gathered}
\end{equation}

где $N$ --- общая длина извлечённого текста.

\section{Вывод}

В данном разделе были приведены результаты проверки смешения типов нумерации иллюстрирующих элементов для приложенного набора файлов и проведена оценка сложности разработанного большими языковыми моделями алгоритма.


\clearpage
