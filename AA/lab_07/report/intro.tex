\ssr{ВВЕДЕНИЕ}
 
Цель --- разработать и реализовать программное обеспечение на языке python для извлечения данных из текстовых файлов, полученных из pdf-файлов с применением библиотеки PyPDF2, с использованием регулярных выражений.

Для достижения поставленной цели нужно было выполнить следующие задачи:
\begin{enumerate}
\item разработать регулярные выражения для проверки наличия смешения типов нумерации (допустимы либо сквозная нумерация, либо пораздельная для всех иллюстрирующих элементов);

\item реализовать функцию для поиска подстроки в pdf-файле с использованием разработанных регулярных выражений;

\item реализовать программное обеспечение, принимающее на вход путь к pdf-файлу, использующее функцию из предыдущего пункта;

\item проверить реализацию на приложенных к лабораторной работе файлах;
\item привести таблицу с колонками <<название использованного pdf-файла>>, <<признак успешного нахождения подстроки>>, <<координаты первого нахождения подстроки>>.

\end{enumerate}

\bigskip

Требуется достичь поставленной цели с использованием не менее трёх больших языковых моделей.

\clearpage
