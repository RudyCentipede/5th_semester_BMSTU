\ssr{ЗАКЛЮЧЕНИЕ}

В результате лабораторной работы было разработано и реализовано программное обеспечение на языке python для извлечения данных из текстовых файлов, полученных из pdf-файлов с применением библиотеки PyPDF2, с использованием регулярных выражений.

Выполнены следующие задачи:
\begin{enumerate}
	\item разработаны регулярные выражения для проверки наличия смешения типов нумерации; 
	\item реализована функция для поиска подстроки в pdf-файле с использованием разработанных регулярных выражений;
	
	\item реализовано программное обеспечение, принимающее на вход путь к pdf-файлу, использующее функцию из предыдущего пункта;
	
	\item проверена реализация на приложенных к лабораторной работе файлах; 
	\item приведена таблица с колонками <<название использованного pdf-файла>>, <<признак успешного нахождения подстроки>>, <<координаты первого нахождения подстроки>>.
\end{enumerate}

\bigskip

Для достижения поставленной цели были использованы три большие языковые модели: GPT-5.2, DeepSeek V3 и Gemini 3 Pro.