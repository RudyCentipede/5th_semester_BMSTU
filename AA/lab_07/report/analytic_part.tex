\chapter{Аналитическая часть}

\section{GPT-5.2}

ChatGPT --- чат-бот с генеративным искусственным
интеллектом, разработанный компанией OpenAI и способный работать в
диалоговом режиме, поддерживающий запросы на естественных языках. Для
тренировки данной языковой модели использовались методы обучения с
учителем и обучения с подкреплением. 

Система способна
отвечать на вопросы, генерировать тексты на разных языках, включая русский,
относящиеся к различным предметным областям. Важной особенностью
является возможность генерации по запросу программ на различных языках
программирования.

По данным OpenAI, GPT-5.2 превзошел или сравнялся с лучшими моделями отрасли в 70,9 \% задач, согласно оценке GDPval, измеряющей определенные задачи интеллектуального труда в 44 профессиях~\cite{article_gpt}.


\section{DeepSeek V3}

Нейросеть DeepSeek --- это современная система искусственного интеллекта, разработанная в Китае, которая позволяет анализировать, обрабатывать и генерировать информацию на основе огромных объёмов данных. Благодаря передовым алгоритмам и инновационным технологиям, DeepSeek способна решать широкий спектр задач: от обработки естественного языка до анализа изображений и прогнозирования различных событий.

DeepSeek V3 --- это новейшая версия китайской нейросети, созданная для максимальной эффективности обработки данных и автоматизации бизнес-процессов. Новая версия отличается повышенной скоростью работы, улучшенной точностью распознавания и адаптивными алгоритмами машинного обучения~\cite{article_deep}.


\section{Gemini 3 Pro}

Gemini 3 Pro способен воплотить в жизнь любую идею благодаря передовым возможностям логического мышления и мультимодальности. Он значительно превосходит 2.5 Pro по всем основным тестам производительности в области искусственного интеллекта.

Она возглавляет таблицу лидеров LMArena с прорывным результатом в 1501 балл по шкале Эло.

Gemini 3 Pro обладает высокими возможностями для решения сложных задач в самых разных областях, таких как наука и математика, с высокой степенью надежности~\cite{article_gem}.

\section{Вывод}

В аналитической части были рассмотрены используемые большие языковые модели.  

\clearpage
