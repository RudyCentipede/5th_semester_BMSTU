\chapter{Технологическая часть}

\section{Средства реализации}

Для реализации алгоритмов был выбран язык C++. Выбор обусловлен тем, что С++ статически типизированный язык программирования, в нём нет сборщика мусора, имеется стандартная библиотека для замера процессорного времени.

Для замера процессорного времени использовалась функция clock() из модуля ctime~\cite{book_iso}.

Для создания нативных потоков использовался класс threads~\cite{book_isopp}.

Был использован шаблон класса lock\_guard для работы с мьютексом и функция захвата/освобождения мьютекса guard()~\cite{book_isopp}.

\section{Реализация алгоритмов}

В листингах~\ref{lst:std} --~\ref{lst:clust} показаны реализации алгоритмов: рекурсивного и нерекурсивного.

\bigskip

\lstinputlisting[label=lst:std, firstline=195, lastline=210,  caption={Реализация последовательного алгоритма DBSCAN}, captionpos=b]{../code/graph_dbscan.cpp}

\clearpage

\lstinputlisting[label=lst:par, firstline=212, lastline=247,  caption={Реализация параллельного алгоритма DBSCAN}, captionpos=b]{../code/graph_dbscan.cpp}

\clearpage

\lstinputlisting[label=lst:clust, firstline=150, lastline=181,  caption={Функция кластеризации}, captionpos=b]{../code/graph_dbscan.cpp}
\clearpage

\section{Функциональные тесты}

В таблице~\ref{table:tests} представлены результаты функционального тестирования реализаций: последовательного алгоритма DBSCAN и параллельного. Каждая реализация каждого алгоритма прошла тесты успешно.
\bigskip

\begin{table}[ht]
	\caption{Результаты функционального тестирования реализаций алгоритма DBSCAN}
	\label{table:tests}
	\centering
	\footnotesize
	\begin{tabular}{|p{2cm}|c|c|p{2.5cm}|p{2.5cm}|}
		\hline
		\textbf{Граф} & \textbf{M} & \textbf{minPts} & \textbf{Ожидаемый результат} & \textbf{Фактический результат} \\
		\hline
		\begin{minipage}{2cm}
			1->2;\\
			2->3;\\
			3->4;\\
			5->6;\\
			6->7;\\
			8;
		\end{minipage} 
		& 2 & 2 & 
		\begin{minipage}{2.5cm}
			clusters: [[1,2,3,4], [5,6,7]]\\
			noise: [8]
		\end{minipage} & 
		\begin{minipage}{2.5cm}
			clusters: [[1,2,3,4], [5,6,7]]\\
			noise: [8]
		\end{minipage} \\
		\hline
		\begin{minipage}{2cm}
			Пусто
		\end{minipage}
		& 1 & 3 & 
		\begin{minipage}{2.5cm}
			clusters: []\\
			noise: []
		\end{minipage} & 
		\begin{minipage}{2.5cm}
			clusters: []\\
			noise: []
		\end{minipage} \\
		\hline
		\begin{minipage}{2cm}
			1;\\
			2;\\
			3;
		\end{minipage}
		& 2 & 2 & 
		\begin{minipage}{2.5cm}
			clusters: []\\
			noise: [1,2,3]
		\end{minipage} & 
		\begin{minipage}{2.5cm}
			clusters: []\\
			noise: [1,2,3]
		\end{minipage} \\
		\hline
		\begin{minipage}{2cm}
			1->2;\\
			2->3;\\
			3->4;\\
			4->5;
		\end{minipage}
		& 3 & 1 & 
		\begin{minipage}{2.5cm}
			clusters: [[1,2,3,4,5]]\\
			noise: []
		\end{minipage} & 
		\begin{minipage}{2.5cm}
			clusters: [[1,2,3,4,5]]\\
			noise: []
		\end{minipage} \\
		\hline
	\end{tabular}
\end{table}


\section{Вывод}

В этом разделе были описаны средства реализации алгоритмов. Также были продемонстрированы листинги реализаций последовательного и параллельного алгоритмов DBSCAN. Приведены результаты функционального тестирования.

\clearpage
