\chapter{Аналитическая часть}

\section{Параллелизм}

Параллелизм --- это одновременное выполнение двух или более операций. В контексте компьютеров, имеется в виду, что одна и та же система выполняет несколько независимых операций параллельно, а не последовательно~\cite{book_parallel1}.

\subsection{Поток}

Поток можно понимать как любой автономный последовательный (линейный) набор команд процессора~\cite{book_unix}.

Источником этого линейного кода для потока могут служить:
\begin{itemize}
	\item бинарный исполняемый файл, на основе которого системой или вызовом группы $spawn()$ запускается новый процесс и создаётся его главный поток;
	\item дубликат кода главного потока процесса родителя при клонировании процессов вызовом $fork()$ (тоже относительно главного потока);
	\item участок кода, оформленный функцией специального типа $(void* ()
	( void* ))$, так называемой функцией потока.
\end{itemize}

\subsection{Многопоточность}

Многопоточность подразумевает разбиение	задач на параллельные потоки выполнения в рамках одного процесса с помощью пользовательских потоков~\cite{book_extr_c}.

Потоки	могут существовать только внутри процессов; не бывает такого потока,	у которого не было бы владельца. Каждый	процесс	содержит по	меньшей	мере один поток, обычно называемый главным или основным~\cite{book_extr_c}.

\subsection{Примитивы синхронизации}

Семафор --- объект, над которым можно провести две атомарные операции: инкремент и декремент внутреннего счётчика -- при условии, что внутренний счётчик не может принимать значение меньше нуля. Если некий
поток пытается уменьшить на единицу значение внутреннего счётчика семафора, значение которого уже равно нулю, то этот поток блокируется до тех пор, пока внутренний счётчик семафора не примет значение, равное 1 или больше (посредством воздействия на него других потоков). Разблокированный поток сможет осуществить декремент нового значения\cite{book_unix}.

\bigskip

Мьютекс (от mutual exclusion -- взаимное исключение) --- это один из
базовых примитивов синхронизации потоков. Этот элемент реализуется на уровне ядра системы и имеет широкий набор атрибутов и настроек. Назначение мьютекса -- защита участка кода от совместного выполнения несколькими потоками. Такой участок кода называют критической секцией, и обычно он является областью модификации общих переменных или обращения к разделяемому ресурсу\cite{book_unix}.

Принцип работы мьютекса заключается в следующем: при обращении
потока к функции блокировки (захвата) проверяется, захвачен ли уже мьютекс, и если да, то вызвавший поток блокируется до освобождения критической секции. Если же нет, то объект мьютекс запоминает, какой поток его захватил (то есть владельца) и устанавливает признак, что он захвачен\cite{book_unix}.

\section{Алгоритм DBSCAN}

Алгоритм DBSCAN (Density Based Spatial Clustering of Applications with Noise), плотностный алгоритм для кластеризации пространственных данных с присутствием шума, был предложен как решение проблемы разбиения (изначально пространственных) данных на кластеры произвольной формы. Большинство алгоритмов, производящих плоское разбиение, создают кластеры по форме близкие к сферическим, так как минимизируют расстояние документов до центра кластера~\cite{book_dbscan}.

\bigskip

Идея, положенная в основу алгоритма, заключается в том, что внутри каждого кластера наблюдается типичная плотность точек (объектов), которая заметно выше, чем плотность снаружи кластера, а также плотность в областях с шумом ниже плотности любого из кластеров. Ещё точнее, что для каждой точки кластера её соседство заданного радиуса должно содержать не менее некоторого числа точек, это число точек задаётся пороговым значением.

\bigskip

\textbf{Определение 1.} Eps соседство точки, обозначаемое как $N_{eps}(p)$, определяется как множество документов, находящихся от точки $p$ на расстояния не более $Eps$: $N_{eps}(p) = {q\in D|dist(p, q) \leqslant Eps}$.
Поиска точек, чьё  $N_{eps}(p)$ содержит хотя бы минимальное число точек $MinPt$ не достаточно, так как точки бывают двух видов: ядровые и граничные.

\textbf{Определение 2.}  Точка $p$ непосредственно плотно достижима из точки $q$ (при заданных $Eps$ и $MinPt$), если $p\in N_{eps}(p)$ и $|N_{eps}(p)|\geqslant MinPt$.

\textbf{Определение 3.} Точка $p$ плотно достижима из точки $q$, если существует последовательность точек $q = p_{1}, p_{2}, ..., p_{n} = p:p_{i+1}$
непосредственно плотно достижимы из $p_{i}$. Это отношение транзитивно, но не симметрично в общем случае, однако симметрично для двух ядровых точек.

\textbf{Определение 4.} Точка $p$ плотно связана с точкой $q$, 
если существует точка $o$: $p$ и $q$ плотно достижимы из $o$ (при заданных $Ep$s и $MinPt$).

\textbf{Определение 5.} Кластер $C_{j}$(при заданных $Eps$ и $MinPt$) -- это не пустое подмножество документов, удовлетворяющее следующих условиям: 
\begin{enumerate}
	\item $\forall p,q:$ если $p\in C_{j}$ и $p$ плотно достижима из $q$ (при заданных $Eps$ и $MinPt$), то $q\in C_{j}$;
	\item $\forall p,q\in C_{j}$: $p$ плотно связана с $q$ (при заданных $Eps$ и $MinPt$). 
\end{enumerate}

Итак, кластер -- это множество плотно связанных точек. В каждом кластере 
содержится хотя бы $MinPt$ документов.

Шум -- это подмножество документов, которые не принадлежат ни одному 
кластеру.

Алгоритм DBSCAN для заданных значений параметров $Eps$ и $MinPt$ исследует кластер следующим образом:
\begin{enumerate}
	\item выбирает случайную точку, являющуюся ядровой, в качестве затравки;
	\item помещает в кластер саму затравку;
	\item помещает в кластер все точки, плотно достижимые из затравки~\cite{book_dbscan}.
\end{enumerate}

\section{Вывод}

В аналитической части были рассмотрены понятия параллелизма и алгоритм DBSCAN.


\clearpage
