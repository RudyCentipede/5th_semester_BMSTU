\chapter{Исследовательская часть}

\section{Технические характеристики}

Замеры процессорного времени проводились на ноутбуке ACER Predator со следующими техническими характеристиками:
\begin{itemize}
	\item процессор: Intel(R) Core(TM) i7-10750H CPU @ 2.60ГГц 2.59 ГГц,
	\item ОЗУ: 16 ГБ,
	\item ОС: Windows 10 Pro 64-разрядная.
\end{itemize}
\bigskip

Во время замеров процессорного времени ноутбук был подключен к электропитанию, сторонними приложениями нагружен не был. 

\section{Замеры процессорного времени}

Были проведены замеры времени работы реализаций алгоритмов для данных, соответствующих лучшему и худшему случаю по трудоёмкости алгоритма Винограда, соответственно когда матрицы на входе имеют чётный размер и нечётный.

Результаты замеров времени для лучшего случая трудоёмкости алгоритма Винограда приведены в таблице~\ref{table:time1}:

\begin{table}[ht]
	\caption{Результаты замеров времени для лучшего случая}
	\label{table:time1}
	\centering
	\begin{tabular}{|c|r@{.}l|r@{.}l|r@{.}l|}
		\hline
		\multicolumn{1}{|c|}{Размер} & \multicolumn{2}{c|}{Стандартный} & \multicolumn{2}{c|}{Винограда} & \multicolumn{2}{c|}{Оптимизированный} \\
		\multicolumn{1}{|c|}{матрицы} & \multicolumn{2}{c|}{алгоритм, мс} & \multicolumn{2}{c|}{алгоритм, мс} & \multicolumn{2}{c|}{алгоритм Винограда, мс} \\
		\hline
		50 & 0&36 & 0&265 & 0&219 \\
		\hline
		100 & 2&828 & 2&016 & 1&687 \\
		\hline
		150 & 9&453 & 6&953 & 5&688 \\
		\hline
		200 & 22&797 & 16&594 & 13&203 \\
		\hline
		250 & 44&875 & 33&203 & 26&422 \\
		\hline
		300 & 81&515 & 61&594 & 48&406 \\
		\hline
		350 & 146&45 & 111&3 & 83&765 \\
		\hline
		400 & 216&95 & 158&95 & 125&39 \\
		\hline
	\end{tabular}
\end{table}

\clearpage

На рисунке~\ref{graph:time1} изображён график зависимости времени работы алгоритмов от размеров матрицы.

\begin{figure}[ht]
\begin{tikzpicture}
	\begin{axis}[
		xlabel={Размер матрицы},
		ylabel={Время, мс},
		legend pos=north west,
		grid=major]
		
		\addplot table[x=Размер, y=Стандартный, col sep=semicolon] {tables/results.csv};
		\addlegendentry{Стандартный}
		
		\addplot table[x=Размер, y=Винограда, col sep=semicolon] {tables/results.csv};
		\addlegendentry{Винограда}
		
		\addplot table[x=Размер, y=Винограда_оптимизированный, col sep=semicolon] {tables/results.csv};
		\addlegendentry{Винограда оптимизированный}
		
	\end{axis}
\end{tikzpicture}
\caption{Сравнение времени работы алгоритмов на чётных размерах матриц}
\label{graph:time1}
\end{figure}

Результаты замеров времени для худшего случая трудоёмкости алгоритма Винограда приведены в таблице~\ref{table:time2}:


\begin{table}[ht]
	\caption{Результаты замеров времени для худшего случая}
	\label{table:time2}
	\centering
	\begin{tabular}{|c|r@{.}l|r@{.}l|r@{.}l|}
		\hline
		\multicolumn{1}{|c|}{Размер} & \multicolumn{2}{c|}{Стандартный} & \multicolumn{2}{c|}{Винограда} & \multicolumn{2}{c|}{Оптимизированный} \\
		\multicolumn{1}{|c|}{матрицы} & \multicolumn{2}{c|}{алгоритм, мс} & \multicolumn{2}{c|}{алгоритм, мс} & \multicolumn{2}{c|}{алгоритм Винограда, мс} \\
		\hline
		51 & 0&39 & 0&281 & 0&25 \\
		\hline
		101 & 2&938 & 2&125 & 1&734 \\
		\hline
		151 & 9&719 & 7&172 & 5&672 \\
		\hline
		201 & 23&219 & 17&015 & 13&656 \\
		\hline
		251 & 51&344 & 38&797 & 30&313 \\
		\hline
		301 & 83&078 & 62&531 & 49&031 \\
		\hline
		351 & 145&42 & 108&78 & 85&265 \\
		\hline
		401 & 219&05 & 163&5 & 126&91 \\
		\hline
	\end{tabular}
\end{table}


На рисунке~\ref{graph:time2} изображён график зависимости времени работы алгоритмов от размеров матрицы.

\clearpage

\begin{figure}[ht]
	\begin{tikzpicture}
		\begin{axis}[
			xlabel={Размер матрицы},
			ylabel={Время, мс},
			legend pos=north west,
			grid=major]
			
			\addplot table[x=Размер, y=Стандартный, col sep=semicolon] {tables/results2.csv};
			\addlegendentry{Стандартный}
			
			\addplot table[x=Размер, y=Винограда, col sep=semicolon] {tables/results.csv};
			\addlegendentry{Винограда}
			
			\addplot table[x=Размер, y=Винограда_оптимизированный, col sep=semicolon] {tables/results.csv};
			\addlegendentry{Винограда оптимизированный}
			
		\end{axis}
	\end{tikzpicture}
	\caption{Сравнение времени работы алгоритмов на нечётных размерах матриц}
	\label{graph:time2}
\end{figure}

Из графиков~\ref{graph:time1} --~\ref{graph:time2} следует, что быстрее всех работает реализация оптимизированного алгоритма Винограда, а медленее всех стандартного алгоритма умножения матриц. Причём оптимизированный алгоритм Винограда быстрее стандартного примерно в 1,7 раз, а алгоритм Винограда без оптимизаций медленее оптимизированного в 1,2 раз. Также алгоритм Винограда на данных, соответствующих лучшему случаю, работает на 6$\%$ быстрее, чем на данных, соответствующих худшему случаю.


\section{Вывод}

В данном разделе были описаны технические характеристики машины, на которой проводились замеры времени. Продемонстрированы результаты замеров процессорного времени, был проведён сравнительный анализ времени работы алгоритмов умножения матриц.


\clearpage
