\chapter{Конструкторская часть}

\section{Функциональные требования}

Разработать программное обеспечение с двумя режимами работы: одиночного выполнения умножения матриц и массированного замера времени выполнения реализаций алгоритмов умножения матриц при варьируемом линейном размере квадратных перемножаемых матриц.

\textbf{Входные данные:}
\begin{itemize}
	\item пункт меню (целое число от 0 до 4),
	\item размеры матрицы,
	\item элементы матрицы.
\end{itemize}

\textbf{Выходные данные:}

Результат умножения матриц и результаты замеров процессорного времени выполнения реализаций алгоритмов умножения матриц.

\section{Схемы алгоритмов}

На рисунках~\ref{alg:std} --~\ref{alg:owino4} представлены схемы алгоритмов умножения матриц.


\begin{figure}[h]
	\centering
	\includegraphics{images/standart.png}
	\caption{Схема стандартного алгоритма умножения матриц}
	\label{alg:std}
\end{figure}
\clearpage

\begin{figure}[h]
	\label{wino1}
	\centering
	\includegraphics[scale=0.65]{images/winograd_1.png}
	\caption{Схема алгоритма Винограда (часть 1)}
\end{figure}
\clearpage

\begin{figure}[h]
	\label{wino2}
	\centering
	\includegraphics[scale=0.8]{images/winograd_2.png}
	\caption{Схема алгоритма Винограда (часть 2)}
\end{figure}
\clearpage

\begin{figure}[h]
	\label{wino3}
	\centering
	\includegraphics[scale=0.8]{images/winograd_3.png}
	\caption{Схема алгоритма Винограда (часть 3)}
\end{figure}
\clearpage

\begin{figure}[h]
	\label{owino1}
	\centering
	\includegraphics[scale=0.8]{images/winograd_opt_1.png}
	\caption{Схема оптимизированного алгоритма Винограда (часть 1)}
\end{figure}
\clearpage

\begin{figure}[h]
	\label{owino2}
	\centering
	\includegraphics[scale=0.9]{images/winograd_opt_2.png}
	\caption{Схема оптимизированного алгоритма Винограда (часть 2)}
\end{figure}
\clearpage

\begin{figure}[h]
	\label{owino3}
	\centering
	\includegraphics[scale=0.7]{images/winograd_opt_3.png}
	\caption{Схема оптимизированного алгоритма Винограда (часть 3)}
\end{figure}
\clearpage

\begin{figure}[h]
	\centering
	\includegraphics[scale=0.7]{images/winograd_opt_4.png}
	\caption{Схема оптимизированного алгоритма Винограда (часть 4)}
	\label{alg:owino4}
\end{figure}
\clearpage

\section{Модель вычислений}

Чтобы вычислить трудоёмкость алгоритмов, введена следующая модель вычислений:

\begin{enumerate}
	\item базовые операции:
	\begin{itemize}
		\item трудоёмкость операций из списка~(\ref{costOne}) равна 1:
		\begin{equation}
			\label{costOne}
			\begin{gathered}
			=, +, \mathrel{+}=, -, \mathrel{-}=, ++, --, ==, \mathrel{!}=, <, <=, >=, >, [], \&\&,\\
			 \&, >>, <<, ||, |
			\end{gathered}
		\end{equation}
		\item трудоёмкость операций из списка~(\ref{costTwo}) равна 2:
		\begin{equation}
			\label{costTwo}
			\begin{gathered}
				*, \mathrel{*}=, /, \mathrel{/}=, \%, \mathrel{\%}=
			\end{gathered}
		\end{equation}
	\end{itemize}
	\item трудоёмкость условного перехода равна 0,
	\item трудоёмкость условного оператора по формуле~(\ref{if}):
	\begin{equation}
		\label{if}
		f_{if} = f_{\text{условия}} + 
		\begin{cases}
			min(f_A, f_B), & \text{-- лучший случай}\\
			max(f_A, f_B) & \text{-- худший случай}
		\end{cases}
	\end{equation}
	\item трудоёмкость цикла по формуле~(\ref{for}):
	\begin{equation}
		\label{for}
		\begin{gathered}
			f_{for} = f_{\text{инициализации}} + f_{\text{сравнения}} + M\cdot(f_{\text{тела}} + f_{\text{инкремента}} + f_{\text{сравнения}})
		\end{gathered}
	\end{equation}
	где $M$ -- число итераций
\end{enumerate}

\section{Трудоёмкость алгоритмов}
\subsection{Стандартный алгоритм умножения матриц}

Трудоёмкость стандартного алгоритма рассчитана по формулам~(\ref{std:first} --~\ref{std:last}):
\begin{equation}
	\label{std:first}
	f_{firstCycle} = 2 + N \cdot (2 + f_{secondCycle})
\end{equation}
\begin{equation}
	f_{secondCycle} = 2 + Q \cdot (2 + f_{thirdCycle})
\end{equation}
\begin{equation}
	f_{thirdCycle} = 2 + M \cdot (2 + f_{body})
\end{equation}
\begin{equation}
	f_{body} = 1 + 8 + 1 + 2 = 12
\end{equation}
\begin{equation}
	\label{std:last}
	f_{algo} = 2 + N \cdot (2 + 2 + Q \cdot (2 + 2 + M \cdot (2 + 12))) = 14NQM + 4NQ + 4N + 2
\end{equation}

Асимптотическая оценка временной сложности алгоритма: $O(14NQM)$. В частном случае квадратных матриц с линейной размерностью N: $O(N^3)$

\subsection{Алгоритм Винограда}

\begin{itemize}
	\item трудоёмкость инициализации массивов mulH и mulV рассчитана по формуле~(\ref{init}):
	\begin{equation}
		\label{init}
		f_{init} = N + M
	\end{equation}
	\item трудоёмкость вычисления mulH рассчитана по формулам~(\ref{mulH:first}) --~(\ref{mulH:last}):
	\begin{equation}
		\label{mulH:first}
		f_{firstCycle} = 2 + N \cdot (2 + f_{secondCycle})
	\end{equation}
	\begin{equation}
		f_{secondCycle} = 4 + \frac{M}{2} \cdot (4 + f_{body})
	\end{equation}
	\begin{equation}
		f_{body} = 1 + 6 + 1 + 2 + 5 = 15
	\end{equation}
	\begin{equation}
		\label{mulH:last}
		f_{mulH} = 2 + N \cdot (2 + 4 + \frac{M}{2} \cdot (4 + 15)) = \frac{19}{2} NM + 6N + 2
	\end{equation}
	\item трудоёмкость вычисления mulV рассчитана по формулам~(\ref{mulV:first}) --~(\ref{mulV:last}):
	\begin{equation}
		\label{mulV:first}
		f_{firstCycle} = 2 + N \cdot (2 + f_{secondCycle})
	\end{equation}
	\begin{equation}
		f_{secondCycle} = 4 + M/2 \cdot (4 + f_{body})
	\end{equation}
	\begin{equation}
		f_{body} = 1 + 6 + 1 + 2 + 5 = 15
	\end{equation}
	\begin{equation}
		\label{mulV:last}
		f_{mulV} = 2 + N \cdot (2 + 4 + \frac{M}{2} \cdot (4 + 15)) = \frac{19}{2} NM + 6N + 2
	\end{equation}

	\item трудоёмкость главного цикла рассчитана по формулам~(\ref{cycle:first}) --~(\ref{cycle:last}):
	\begin{equation}
		\label{cycle:first}
		f_{firstCycle} = 2 + N \cdot (2 + f_{secondCycle})
	\end{equation}
	\begin{equation}
		f_{secondCycle} = 2 + Q \cdot (2 + 7 + f_{thirdCycle})
	\end{equation}
	\begin{equation}
		f_{thirdCycle} = 4 + \frac{M}{2} \cdot (4 + f_{body})
	\end{equation}
	\begin{equation}
		f_{body} = 12 + 3 + 2 + 1 + 10 = 28
	\end{equation}
	\begin{equation}
		\label{cycle:last}
		\begin{gathered}
		f_{cycle} = 2 + N \cdot (2 + 2 + Q \cdot (2 + 7 + 4 + \frac{M}{2} \cdot (4 + 28))) = \\ = 16NQM + 13NQ + 4N + 2
		\end{gathered}
	\end{equation}
	\item трудоёмкость цикла, нужного для подсчёта значений при нечётном размере матрицы, расчитана по формуле~(\ref{odd}):
	\begin{equation}
		\label{odd}
		f_{odd} = 3 + 
		\begin{cases}
			2 + N \cdot (2 + 2 + Q \cdot (2 + 14)), & \text{-- нечётный размер} \\
			0, & \text{-- иначе}
		\end{cases}
	\end{equation}
\end{itemize}

Трудоёмкость алгоритма Винограда для худшего случая, когда у матрицы нечётный размер, рассчитана по формуле~(\ref{worst}):
\begin{equation}
	\label{worst}
	\begin{gathered}
	f_{algoWorst} = f_{init} + f_{mulH} + f_{mulV} + f{cycle} + f_{odd} = \\ = N + M + \frac{19}{2} NM + 6N + 2 + \frac{19}{2} NM + 6N + 2 + 16NQM +\\ + 13NQ + 4N + 2 + 3 + 2 + N \cdot (2 + 2 + Q \cdot (2 + 14))= \\ = 16NQM + 29NQ + 19NM + 21N + M + 11
	\end{gathered}
\end{equation}

Трудоёмкость алгоритма Винограда для лучшего случая, когда у матрицы чётный размер, рассчитана по формуле~(\ref{best}):
\begin{equation}
	\label{best}
	\begin{gathered}
		f_{algoBest} = f_{init} + f_{mulH} + f_{mulV} + f{cycle} + f_{odd} = \\ = N + M + \frac{19}{2} NM + 6N + 2 + \frac{19}{2} NM + 6N + 2 + 16NQM +\\ + 13NQ + 4N + 2 + 3 = 16NQM + 13NQ + 19NM + 17N + M + 9
	\end{gathered}
\end{equation}

Асимптотическая оценка временной сложности алгоритма для худшего и лучшего случая: $O(16NQM)$. В частном случае квадратных матриц с линейной размерностью N: $O(N^3)$

\subsection{Оптимизированный алгоритм Винограда}

\begin{itemize}
	\item трудоёмкость инициализации массивов mulH и mulV рассчитана по формуле~(\ref{o:init}):
	\begin{equation}
		\label{o:init}
		f_{init} = N + M
	\end{equation}
	
	\item трудоёмкость предварительного вычисления $\frac{M}{2}$ равна 3,
	\item трудоёмкость вычисления mulH рассчитана по формулам~(\ref{o:mulH:first}) --~(\ref{o:mulH:last}):
	\begin{equation}
		\label{o:mulH:first}
		f_{firstCycle} = 2 + N \cdot (5 + f_{secondCycle})
	\end{equation}
	\begin{equation}
		f_{secondCycle} = 2 + \frac{M}{2} \cdot (2 + f_{body})
	\end{equation}
	\begin{equation}
		f_{body} = 1 + 4 + 1 + 2 + 5 = 13
	\end{equation}
	\begin{equation}
		\label{o:mulH:last}
		f_{mulH} = 2 + N \cdot (5 + 2 + \frac{M}{2} \cdot (2 + 13)) = \frac{15}{2} NM + 7N + 2
	\end{equation}
	\item трудоёмкость вычисления mulV рассчитана по формулам~(\ref{o:mulV:first} --~(\ref{o:mulV:last}):
	\begin{equation}
		\label{o:mulV:first}
		f_{firstCycle} = 2 + N \cdot (5 + f_{secondCycle})
	\end{equation}
	\begin{equation}
		f_{secondCycle} = 2 + \frac{M}{2} \cdot (2 + f_{body})
	\end{equation}
	\begin{equation}
		f_{body} = 1 + 4 + 1 + 2 + 5 = 13
	\end{equation}
	\begin{equation}
		\label{o:mulV:last}
		f_{mulV} = 2 + N \cdot (5 + 2 + \frac{M}{2} \cdot (2 + 13)) = \frac{15}{2} NM + 7N + 2
	\end{equation}
	
	\item трудоёмкость главного цикла рассчитана по формулам~(\ref{o:cycle:first} --~(\ref{o:cycle:last}):
	\begin{equation}
		\label{o:cycle:first}
		f_{firstCycle} = 2 + N \cdot (2 + 4 + f_{secondCycle})
	\end{equation}
	\begin{equation}
		f_{secondCycle} = 2 + Q \cdot (2 + 7 + f_{thirdCycle})
	\end{equation}
	\begin{equation}
		f_{thirdCycle} = 2 + \frac{M}{2} \cdot (2 + 3 + f_{body})
	\end{equation}
	\begin{equation}
		f_{body} = 4 + 2 + 6 + 2 = 14
	\end{equation}
	\begin{equation}
		\label{o:cycle:last}
		\begin{gathered}
			f_{cycle} = 2 + N \cdot (6 + 2 + Q \cdot (2 + 7 + 2 + \frac{M}{2} \cdot (5 + 14))) = \\ = \frac{19}{2} NQM + 11NQ + 8N + 2
		\end{gathered}
	\end{equation}
	\item трудоёмкость цикла, нужного для подсчёта значений при нечётном размере матрицы, расчитана по формуле~(\ref{o:odd}):
	\begin{equation}
		\label{o:odd}
		f_{odd} = 3 + 
		\begin{cases}
			4 + N \cdot (2 + 7 + 2 + Q \cdot (2 + 8)), & \text{-- нечётный размер} \\
			0, & \text{-- иначе}
		\end{cases}
	\end{equation}
\end{itemize}

Трудоёмкость оптимизированного алгоритма Винограда для худшего случая, когда у матрицы нечётный размер, рассчитана по формуле~(\ref{o:worst}):
\begin{equation}
	\label{o:worst}
	\begin{gathered}
		f_{algoWorst} = f_{init} + f_{mulH} + f_{mulV} + f{cycle} + f_{odd} = \\ = N + M + \frac{15}{2} NM + 7N + 2 + \frac{15}{2} NM + 7N + 2 + \frac{19}{2}NQM +\\ + 11NQ + 8N + 2 + 3 + 4 + N \cdot (2 + 7 + 2 + Q \cdot (2 + 8))= \\ = \frac{19}{2}NQM + 21NQ + 15NM + 34N + M + 13
	\end{gathered}
\end{equation}

Трудоёмкость оптимизированного алгоритма Винограда для лучшего случая, когда у матрицы чётный размер, рассчитана по формуле~(\ref{o:best}):
\begin{equation}
	\label{o:best}
	\begin{gathered}
		f_{algoBest} = f_{init} + f_{mulH} + f_{mulV} + f{cycle} + f_{odd} = \\ = N + M + \frac{15}{2} NM + 7N + 2 + \frac{15}{2} NM + 7N + 2 + \frac{19}{2}NQM +\\ + 11NQ + 8N + 2 + 3 = \frac{19}{2}NQM + 11NQ + 15NM + 23N + M + 9
	\end{gathered}
\end{equation}

Асимптотическая оценка временной сложности алгоритма для худшего и лучшего случая: $O(\frac{19}{2}NQM)$. В частном случае квадратных матриц с линейной размерностью N: $O(N^3)$
\clearpage

\section{Вывод}

В данном разделе были описаны функциональные требования к программе, построены схемы алгоритмов умножения матриц: стандартного, алгоритма Винограда и оптимизированного алгоритма Винограда. 

Была введена модель вычислений, в соответствии с которой были рассчитаны трудоёмкости алгоритмов умножения матриц. Проведённая оценка трудоемкости алгоритмов показала, что трудоёмкость выполнения алгоритма Винограда в случае оптимизации в 1.68 раз меньше, чем у простого алгоритма Винограда.

\clearpage
