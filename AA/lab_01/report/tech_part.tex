\chapter{Технологическая часть}

\section{Средства реализации}

Для реализации алгоритмов был выьран язык C. Выбор обусловлен тем, что С -- статически типизированный язык программирования, в нём нет сборщика мусора и имеется стандартная библиотека для замера процессорного времени.

Для замера процессорного времени использовалась функция clock() из модуля time.~\cite{book_iso}

\section{Реализация алгоритмов}

В листингах~(\ref{lst:std} -~\ref{lst:owino}) показаны реализации алгоритмов умножения матриц: стандартного, Винограда и оптимизированного алгоритма Винограда.

\bigskip

\lstinputlisting[label=lst:std, firstline=42, lastline=55,  caption={Реализация стандартного алгоритма умножения матриц}, captionpos=b]{../code/matrix.c}

\clearpage

\lstinputlisting[label=lst:wino, firstline=57, lastline=90,  caption={Реализация алгоритма Винограда}, captionpos=b]{../code/matrix.c}

\clearpage

\lstinputlisting[label=lst:owino, firstline=92, lastline=145,  caption={Реализация оптимизированного алгоритма Винограда}, captionpos=b]{../code/matrix.c}
\clearpage

\section{Функциональные тесты}

В таблице~\ref{table:tests} представлены результаты функционального тестирования реализаций: стандартного алгоритма умножения матриц, алгоритма Винограда и оптимизированного алгоритма Винограда. Каждая реализация каждого алгоритма прошла тесты успешно.
\bigskip

\begin{table}[ht]
	\caption{Результаты функционального тестирования}
	\label{table:tests}
\begin{adjustbox}{max width=\textwidth}

	\begin{tabular}{|c|c|c|c|}
		\hline
		\multicolumn{2}{|c|}{\textbf{Входные данные}} & \multicolumn{2}{c|}{\textbf{Результат}} \\
		\hline
		\textbf{Матрица A} & \textbf{Матрица B} & \textbf{Ожидаемый} & \textbf{Фактический} \\
		\hline
		$\begin{pmatrix} 
			1 & 2 \\ 
			3 & 4 
		\end{pmatrix}
		$ & $()$ & \text{Ошибка} & \text{Ошибка} \\
		\hline
		$\begin{pmatrix} 
			1 & 2 & 3 \\ 
			4 & 5 & 6 \\ 
			7 & 8 & 9 
		\end{pmatrix}
		$ & $(1, 2, 3)$ & \text{Ошибка} & \text{Ошибка} \\
		\hline
		$\begin{pmatrix} 
			1 & 2 & 3 \\ 
			4 & 5 & 6 \\ 
			7 & 8 & 9 
		\end{pmatrix}
		$ & $
		\begin{pmatrix} 
			1 & 0 & 0 \\ 
			0 & 1 & 0 \\ 
			0 & 0 & 1 
		\end{pmatrix}
		$ & $
		\begin{pmatrix} 
			1 & 2 & 3 \\ 
			4 & 5 & 6 \\ 
			7 & 8 & 9 
		\end{pmatrix}
		$ & $
		\begin{pmatrix} 
			1 & 2 & 3 \\ 
			4 & 5 & 6 \\ 
			7 & 8 & 9 
		\end{pmatrix}$ \\
		\hline
		$\begin{pmatrix} 
			1 & 2 \\ 
			3 & 4 \\ 
			5 & 6 
		\end{pmatrix}
		$ & $
		\begin{pmatrix} 
			1 & 2 & 3 \\
			4 & 5 & 6 
		\end{pmatrix}
		$ & $
		\begin{pmatrix} 
			9 & 12 & 15 \\ 
			19 & 26 & 33 \\ 
			29 & 40 & 51 
		\end{pmatrix}
		$ & $
		\begin{pmatrix} 
			9 & 12 & 15 \\ 
			19 & 26 & 33 \\ 
			29 & 40 & 51 
		\end{pmatrix}$ \\
		\hline
	\end{tabular}
\end{adjustbox}
\end{table}


\section{Вывод}

В этом разделе были описаны средства реализации алгоритмов. Также были продемонстрированы листинги реализаций: стандартного алгоритма умножения матриц, алгоритма Винограда и оптимизированного алгоритма Винограда. Приведены результаты функционального тестирования.

\clearpage
