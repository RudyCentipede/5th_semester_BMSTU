\ssr{ЗАКЛЮЧЕНИЕ}

В данной лабораторной работе был проведён сравнительный анализ рекурсивного и нерекурсивного алгоритмов решения задачи определения количества единиц в последовательности, состоящей из нулей и единиц, заканчивающейся числом 2.

Выполнены задачи:
\begin{enumerate}
	\item разработаны рекурсивный и нерекурсивный алгоритмы решения задачи,
	\item описаны средства разработки и инструменты замера 
	процессорного времени выполнения реализации алгоритмов;
	\item реализованы разработанные алгоритмы;
	\item выполнено тестирование реализации алгоритмов;
	\item выполнена теоретическая оценка затрачиваемой 
	реализацией каждого алгоритма памяти (для рекурсивного 
	алгоритма на материале анализа высоты дерева 
	рекурсивных вызовов и оценки затрачиваемой на один 
	вызов функции памяти);
	\item выполнены замеры процессорного времени выполнения 
	реализации алгоритмов в зависимости от варьируемого 
	входа;
	\item оценена трудоёмкость двух алгоритмов в худшем случае;
	\item сравнены результаты замеров процессорного времени и 
	оценки трудоёмкости;
	\item  сделаны выводы из сравнительного анализа реализации 
	рекурсивного и нерекурсивного алгоритмов решения одной и 
	той же задачи по критериям ёмкостной эффективности (на материале теоретической оценки пиковой временной эффективности на материале результатов измерений).
	
\end{enumerate}

\bigskip

В результате лабораторной работы было выявлено, что реализация нерекурсивного алгоритма быстрее реализации рекурсивного примерно в 3.2 раза. Для работы рекурсивного алгоритма требуется больше памяти, чем для работы нерекурсивного алгоритма: асимптотика рекурсивного алгоритма -- $O(n)$, а нерекурсивного -- $O(1)$.
	