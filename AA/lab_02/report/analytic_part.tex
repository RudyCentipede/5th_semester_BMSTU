\chapter{Аналитическая часть}

\section{Рекурсия}

В математике и программировании рекурсия -- это метод определения или выражения
функции или процедуры посредством той же функции и процедуры. Рекурсию обычно рассматривают в качестве антипода итерации. Соответственно различают два больших класса
алгоритмов: итерационные и рекурсивные.

В основе итерационных алгоритмов лежит итерация -- многократное повторение одних и тех же действий.
Рекурсивный алгоритм -- это алгоритм, определяемый через себя. В основе рекурсивных алгоритмов лежит рекурсия.~\cite{book_rec1} 

В контексте программирования рекурсивной называется функция, которая вызывает сама себя.~\cite{book_rec2} 

\section{Вывод}

В аналитической части были рассмотрены понятия рекурсии, итерационного и рекурсивного алгоритмов, рекурсивной функции.


\clearpage
