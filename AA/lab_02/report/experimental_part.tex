\chapter{Исследовательская часть}

\section{Технические характеристики}

Замеры процессорного времени проводились на ноутбуке ACER Predator со следующими техническими характеристиками:
\begin{itemize}
	\item процессор: Intel(R) Core(TM) i7-10750H CPU @ 2.60ГГц 2.59 ГГц,
	\item ОЗУ: 16 ГБ,
	\item ОС: Windows 10 Pro 64-разрядная.
\end{itemize}
\bigskip

Во время замеров процессорного времени ноутбук был подключен к электропитанию, сторонними приложениями нагружен не был. 

\section{Замеры процессорного времени}

Были проведены замеры времени работы реализаций алгоритмов для данных, соответствующих худшему случаю: когда последовательность полностью состоит из единиц.

Результаты замеров процессорного времени приведены в таблице~\ref{table:time}:

\bigskip

\begin{table}[ht]
	\caption{Результаты замеров времени работы реализаций алгоритмов подсчёта единиц}
	\label{table:time}
	\centering
	\begin{tabular}{|c|r@{.}l|r@{.}l|}
		\hline
		\multicolumn{1}{|c|}{Длина} & \multicolumn{2}{c|}{Рекурсивный} & \multicolumn{2}{c|}{Нерекурсивный} \\
		\multicolumn{1}{|c|}{последовательности} & \multicolumn{2}{c|}{алгоритм, нс} & \multicolumn{2}{c|}{алгоритм, нс} \\
		\hline
		1000 & 6431&51 & 1879&79 \\
		\hline
		2000 & 12415&00 & 3916&74 \\
		\hline
		3000 & 19764&65 & 6402&30 \\
		\hline
		4000 & 25434&50 & 8060&47 \\
		\hline
		5000 & 30984&43 & 9787&86 \\
		\hline
		6000 & 37403&94 & 11666&33 \\
		\hline
		7000 & 44202&55 & 13603&21 \\
		\hline
		8000 & 50720&17 & 15071&54 \\
		\hline
		9000 & 57141&32 & 17016&54 \\
		\hline
		10000 & 64082&05 & 19018&39 \\
		\hline
	\end{tabular}
\end{table}

\clearpage

На рисунке~\ref{graph:time} изображён график зависимости времени работы алгоритмов от длины последовательности.

\bigskip

\begin{figure}[ht]
	\begin{tikzpicture}
		\begin{axis}[
			width=0.9\textwidth,
			height=0.7\textwidth,
			xlabel={Длина последовательности},
			ylabel={Время, нс},
			legend pos=north west,
			grid=major,
			xtick={1000,2000,3000,4000,5000,6000,7000,8000,9000,10000},
			xticklabels={0.1$\times 10^4$,0.2$\times 10^4$,0.3$\times 10^4$,0.4$\times 10^4$,0.5$\times 10^4$,0.6$\times 10^4$,0.7$\times 10^4$,0.8$\times 10^4$,0.9$\times 10^4$,1.0$\times 10^4$},
			ytick={0,10000,20000,30000,40000,50000,60000,70000},
			yticklabels={0$\times 10^4$,1$\times 10^4$,2$\times 10^4$,3$\times 10^4$,4$\times 10^4$,5$\times 10^4$,6$\times 10^4$,7$\times 10^4$},
			x tick label style={rotate=45, anchor=east, font=\small},
			xlabel style={yshift=5mm},
			ylabel style={yshift=-3mm},
			every axis x label/.style={at={(0.5,-0.15)},anchor=north},
			every axis y label/.style={at={(-0.1,0.5)},anchor=south,rotate=90},
			scaled x ticks=false,
			scaled y ticks=false
			]
			
			\addplot table[x=Длина, y=Рекурсивный, col sep=semicolon] {tables/results.csv};
			\addlegendentry{Рекурсивный}
			
			\addplot table[x=Длина, y=Нерекурсивный, col sep=semicolon] {tables/results.csv};
			\addlegendentry{Нерекурсивный}
			
		\end{axis}
	\end{tikzpicture}
	\caption{Сравнение времени работы рекурсивного и нерекурсивного алгоритмов}
	\label{graph:time}
\end{figure}


Из графика~\ref{graph:time} следует, что реализация нерекурсивного алгоритма работает быстрее реализации рекурсивного алгоритма примерно в 3.2 раза.

\section{Теоретическая оценка затрачиваемой памяти}

Размеры типов данных в байтах:
\begin{itemize}
	\item $S_{reg} = 8$ -- размер регистра,
	\item $S_{ret} = 8$ -- размер адреса возврата,
	\item $S_{int} = 8$ -- размер типа $int$.
\end{itemize}

\subsection{Рекурсивный алгоритм}

Память, затрачиваемая реализацией рекурсивного алгоритма, в байтах рассчитана по формулам~(\ref{mem:h} --~\ref{mem:total}):

\begin{itemize}
	\item глубина дерева рекурсивных вызовов рассчитана по формуле~(\ref{mem:h}):
	\begin{equation}
		\label{mem:h}
		R(D) = N,
	\end{equation}
	
	\item память в байтах, затраченная на один рекурсивный вызов, рассчитана по формуле~(\ref{mem:rec}):
	\begin{equation}
		\label{mem:rec}
		M_{reccall} = 19\cdot S_{reg} + S_{ret} = 160,
	\end{equation}
	
	\item память в байтах, затраченная на локальные переменные, рассчитана по формуле~(\ref{mem:local}):
	\begin{equation}
		\label{mem:local}
		M_{local} = 0,
	\end{equation}
	
	\item общая память в байтах для рекурсивного алгоритма рассчитана по формуле~(\ref{mem:total}):
	\begin{equation}
		\label{mem:total}
		M_{total} = (M_{reccall} + M_{local})\cdot (R(D) + 1) = 160N + 160.
	\end{equation}
	
\end{itemize}

\subsection{Нерекурсивный алгоритм}

Память, затрачиваемая реализацией нерекурсивного алгоритма, в байтах рассчитана по формулам~(\ref{mem:iter} --~\ref{mem:i:total}):

\begin{itemize}
	
	\item память в байтах, затраченная на вызов функции, рассчитана по формуле~(\ref{mem:iter}):
	\begin{equation}
		\label{mem:iter}
		M_{itercall} = 19\cdot S_{reg} + S_{ret} = 160,
	\end{equation}
	
	\item память в байтах, затраченная на локальные переменные, рассчитана по формуле~(\ref{mem:i:local}):
	\begin{equation}
		\label{mem:i:local}
		M_{local} = 2\cdot S_{int} = 2\cdot 8 = 16,
	\end{equation}
	
	\item общая память в байтах для нерекурсивного алгоритма рассчитана по формуле~(\ref{mem:i:total}):
	\begin{equation}
		\label{mem:i:total}
		M_{total} = M_{itercall} + M_{local} = 160 + 16 = 176.
	\end{equation}
	
\end{itemize}

\section{Вывод}

В данном разделе были описаны технические характеристики машины, на которой проводились замеры времени. Продемонстрированы результаты замеров процессорного времени, был проведён сравнительный анализ времени работы алгоритмов умножения матриц.

Также была проведена теоретическая оценка затрачиваемой реализациями алгоритмов памяти. Для работы рекурсивного алгоритма требуется больше памяти, чем для работы нерекурсивного алгоритма: асимптотика рекурсивного алгоритма -- $O(n)$, а нерекурсивного -- $O(1)$.


\clearpage
