\ssr{ВВЕДЕНИЕ}
 
 Целью данной лабораторной работы был сравнительный анализ рекурсивного и нерекурсивного алгоритмов решения задачи определения количества единиц в последовательности, состоящей из нулей и единиц, заканчивающейся числом 2.
 
 Для достижения поставленной цели необходимо было выполнить следующие задачи:
 \begin{enumerate}
 	\item разработать рекурсивный и нерекурсивный алгоритмы решения задачи;
 	\item описать средства разработки и инструменты замера 
 	процессорного времени выполнения реализации алгоритмов;
 	\item реализовать разработанные алгоритмы;
 	\item выполнить тестирование реализации алгоритмов;
 	\item выполнить теоретическую оценку затрачиваемой 
 	реализацией каждого алгоритма памяти (для рекурсивного 
 	алгоритма на материале анализа высоты дерева 
 	рекурсивных вызовов и оценки затрачиваемой на один 
 	вызов функции памяти);
 	\item выполнить замеры процессорного времени выполнения 
 	реализации алгоритмов в зависимости от варьируемого 
 	входа;
 	\item оценить трудоёмкость двух алгоритмов в худшем случае;
 	\item сравнить результаты замеров процессорного времени и 
 	оценки трудоёмкости;
 	\item  сделать выводы из сравнительного анализа реализации 
 	рекурсивного и нерекурсивного алгоритмов решения одной и 
 	той же задачи по критериям ёмкостной эффективности (на материале теоретической оценки пиковой временной эффективности на материале результатов измерений).
 	
\end{enumerate}

\clearpage
