\ssr{ВВЕДЕНИЕ}
 
 Цель: на материале графовых моделей алгоритмов выделить участки программ, которые могут быть исполнены параллельно.
 
 Для достижения поставленной цели необходимо было выполнить следующие задачи:
 \begin{enumerate}
 	\item описать два алгоритма (рекурсивный и нерекурсивный) решения задачи определения количества единиц в последовательности, состоящей из нулей и единиц, заканчивающейся числом 2;
 	\item описать реализации алгоритмов четырьмя графовыми моделями --- графом управления, информационным графом, операционной историей, 
 	информационной историей;
 	\item указать участки каждой программы, которые могут быть 
 	исполнены параллельно, или отсутствие таковых. 
\end{enumerate}

\clearpage
