\chapter{Аналитическая часть}

\section{Рекурсия}

В математике и программировании рекурсия -- это метод определения или выражения
функции или процедуры посредством той же функции и процедуры. Рекурсию обычно рассматривают в качестве антипода итерации. Соответственно различают два больших класса
алгоритмов: итерационные и рекурсивные.

В основе итерационных алгоритмов лежит итерация -- многократное повторение одних и тех же действий.
Рекурсивный алгоритм -- это алгоритм, определяемый через себя. В основе рекурсивных алгоритмов лежит рекурсия.~\cite{book_rec1} 

В контексте программирования рекурсивной называется функция, которая вызывает сама себя.~\cite{book_rec2} 


\section{Графовые модели программ}

Операционное отношение -- это отношение между операторами программы, которое определяется фактом выполнения одного оператора непосредственно за другим. Оно задаёт порядок передачи управления между операторами в программе.~\cite{book_parallel}

Информационное отношение -- это отношение между операторами программы, при котором один оператор использует в качестве аргументов результаты выполнения других операторов. Оно определяет зависимость по данным между операторами.~\cite{book_parallel}

Операционная история -- это ориентированный граф, который строится на основе наблюдения за выполнением программы на конкретных входных данных. Вершины соответствуют каждому срабатыванию операторов, а дуги соединяют вершины в соответствии с операционными отношениями.~\cite{book_parallel}

Информационная история -- это ориентированный граф, который строится на основе наблюдения за выполнением программы на конкретных входных данных. Вершины соответствуют каждому срабатыванию операторов, а дуги соединяют вершины в соответствии с информационными отношениями.~\cite{book_parallel}

Информационный граф -- это модель программы, не зависящая от входных данных. Вершины графа соответствуют операторам исходной программы. Две вершины соединяются дугой, если между какими-либо срабатываниями соответствующих операторов теоретически возможно информационное отношение.~\cite{book_parallel}

Управляющий граф --  это модель программы, не зависящая от входных данных. Вершины графа соответствуют операторам исходной программы. Две вершины соединяются дугой, если текст программы допускает выполнение одного оператора непосредственно за другим, то есть возможно операционное отношение.~\cite{book_parallel}

\section{Вывод}

В аналитической части были рассмотрены понятия рекурсии, итерационного и рекурсивного алгоритмов, рекурсивной функции. Рассмотрены основные понятия графовых моделей программ.


\clearpage
