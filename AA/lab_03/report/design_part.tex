\chapter{Конструкторская часть}

\section{Функциональные требования}

Разработать программное обеспечение для определения количества единиц в последовательности, состоящей из нулей и единиц, заканчивающейся числом 2.

\textbf{Входные данные:}
\begin{itemize}
	\item пункт меню (целое число от 0 до 3),
	\item элементы последовательности -- целые числа от 0 до 2.
\end{itemize}

\textbf{Выходные данные:}

Целое число -- количество единиц в последовательности.

\section{Схемы алгоритмов}

На рисунках~\ref{alg:iter} --~\ref{alg:rec} представлены схемы алгоритмов.


\begin{figure}[h]
	\centering
	\includegraphics[scale=0.9]{images/iter.pdf}
	\caption{Схема нерекурсивного алгоритма}
	\label{alg:iter}
\end{figure}
\clearpage

\begin{figure}[h]
	\centering
	\includegraphics[scale=0.9]{images/recursive.pdf}
	\caption{Схема рекурсивного алгоритма}
	\label{alg:rec}
\end{figure}
\clearpage




\section{Вывод}

В данном разделе были описаны функциональные требования к программе, построены схемы алгоритмов: нерекурсивного и рекурсивного. 
\clearpage
