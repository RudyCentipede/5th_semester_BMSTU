\chapter{Технологическая часть}

\section{Средства реализации}

Для реализации алгоритмов был выбран язык C. Выбор обусловлен тем, что С -- статически типизированный язык программирования, в нём нет сборщика мусора и имеется стандартная библиотека для замера процессорного времени.

Для замера процессорного времени использовалась функция clock() из модуля time.~\cite{book_iso}

\section{Реализация алгоритмов}

В листингах~(\ref{lst:rec} -~\ref{lst:iter}) показаны реализации алгоритмов: рекурсивного и нерекурсивного.

\bigskip

\lstinputlisting[label=lst:rec, firstline=3, lastline=11,  caption={Реализация рекурсивного алгоритма}, captionpos=b]{../code/algorithms.cpp}

\lstinputlisting[label=lst:iter, firstline=13, lastline=20,  caption={Реализация нерекурсивного алгоритма}, captionpos=b]{../code/algorithms.cpp}

\clearpage

\section{Функциональные тесты}

В таблице~\ref{table:tests} представлены результаты функционального тестирования реализаций: рекурсивного и нерекурсивного алгоритмов. Каждая реализация каждого алгоритма прошла тесты успешно.
\bigskip

\begin{table}[ht]
	\caption{Результаты функционального тестирования алгоритмов подсчёта единиц}
	\label{table:tests}
	\begin{adjustbox}{max width=\textwidth}
		\begin{tabular}{|c|c|c|c|}
			\hline
			\multicolumn{2}{|c|}{\textbf{Входные данные}} & \multicolumn{2}{c|}{\textbf{Результат}} \\
			\hline
			\textbf{Последовательность} & \textbf{Завершающий символ} & \textbf{Ожидаемый} & \textbf{Фактический} \\
			\hline
			($1, 0, 1, 1, 0$) & 2 & 3 & 3 \\
			\hline
			($1, 1, 1, 1, 1$) & 2 & 5 & 5 \\
			\hline
			($0, 0, 0, 0, 0$) & 2 & 0 & 0 \\
			\hline
			($1$) & 2 & 1 & 1 \\
			\hline
			($0$) & 2 & 0 & 0 \\
			\hline
			() & 2 & 0 & 0 \\
			\hline
			($2$) & 2 & 0 & 0 \\
			\hline
			($1, 0, 1, 0, 1$) & - & \text{Ошибка} & \text{Ошибка} \\
			\hline
			($1, 0, 0, 1$) & 3 & \text{Ошибка} & \text{Ошибка} \\
			\hline
		\end{tabular}
	\end{adjustbox}
\end{table}


\section{Вывод}

В этом разделе были описаны средства реализации алгоритмов. Также были продемонстрированы листинги реализаций алгоритмов: рекурсивного и нерекурсивного. Приведены результаты функционального тестирования.

\clearpage
