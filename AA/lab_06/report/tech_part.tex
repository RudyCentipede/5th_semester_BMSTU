\chapter{Технологическая часть}

\section{Средства реализации}

Для реализации алгоритмов был выбран язык C++17~\cite{book_isopp}. Выбор обусловлен тем, что C++ имеет все необходимые стандартные библиотеки для решения задачи.

\section{Реализация алгоритмов}

В листингах~\ref{lst:bruteforce} --~\ref{lst:ant} показаны реализации алгоритмов: полного перебора и муравьиного.

\clearpage

\begin{lstlisting}[
	language=C++,
	label=lst:bruteforce,
	caption={Реализация алгоритма полного перебора},
	captionpos=b
	]
	void solve_bruteforce_iterative() {
		int perm[MAX_N];
		for (int i = 0; i < n; ++i)
			perm[i] = i;
		bestLenBrute = INF;
		bool first = true;
		do {
			double len = 0.0;
			bool ok = true;
			for (int i = 0; i < n - 1; ++i) {
				int a = perm[i];
				int b = perm[i + 1];
				
				if (distM[a][b] >= INF / 2.0) {
					ok = false;
					break;
				}
				len += distM[a][b];
				if (len >= bestLenBrute) {
					ok = false;
					break;
				}
			}
			if (ok) {
				if (first || len < bestLenBrute) {
					first = false;
					bestLenBrute = len;
					memcpy(bestPathBrute, perm, n * sizeof(int));
				}
			}
		} while (std::next_permutation(perm, perm + n));
	}
\end{lstlisting}


\clearpage

\begin{lstlisting}[
	language=C++,
	label=lst:ant,
	caption={Реализация муравьиного алгоритма},
	captionpos=b
	]
	void run_aco() {
		init_heuristic();
		init_pheromone();
		bestLenACO = INF;
		for (int iter = 0; iter < MAX_ITER; ++iter) {
			for (int k = 0; k < NUM_ANTS; ++k)
				generate_ant_solution(k);
			
			int bestAntIter = -1;
			double bestLenIter = INF;
			
			for (int k = 0; k < NUM_ANTS; ++k) {
				if (antLen[k] < bestLenIter) {
					bestLenIter = antLen[k];
					bestAntIter = k;
				}
			}
			
			if (bestLenIter < bestLenACO) {
				bestLenACO = bestLenIter;
				for (int i = 0; i < n; ++i)
				bestPathACO[i] = antPath[bestAntIter][i];
			}
			
			for (int i = 0; i < n; ++i)
				for (int j = 0; j < n; ++j)
					pheromone[i][j] *= (1.0 - RHO);
			
			for (int k = 0; k < NUM_ANTS; ++k) {
				if (antLen[k] >= INF / 2.0)
					continue;
			
				double contrib = QVAL / antLen[k];
				for (int s = 0; s < n - 1; ++s) {
					int i = antPath[k][s];
					int j = antPath[k][s + 1];
					pheromone[i][j] += contrib;
				}
			}
			if (bestLenACO < INF / 2.0) {
				double eliteContrib = ELITE_ANTS * QVAL / bestLenACO;
				for (int s = 0; s < n - 1; ++s) {
					int i = bestPathACO[s];
					int j = bestPathACO[s + 1];
					pheromone[i][j] += eliteContrib;
				}
			}
		}
	}
\end{lstlisting}

\clearpage

%\lstinputlisting[label=lst:clust, firstline=150, lastline=181,  caption={Функция кластеризации}, captionpos=b]{../code/graph_dbscan.cpp}
%\clearpage

\section{Тестирование}

В таблице~\ref{table:tests} представлены результаты тестирования реализации алгоритма полного перебора. Все тесты пройдены успешно.
\bigskip

\newcolumntype{C}[1]{>{\centering\arraybackslash}m{#1}}
\setlength{\extrarowheight}{3pt}
\renewcommand{\arraystretch}{1.2} 

\newcolumntype{C}[1]{>{\centering\arraybackslash}m{#1}}

\begin{table}[h]
	\centering
	\caption{Результаты тестирования алгоритма полного перебора}
	\label{table:tests}
	\setlength{\tabcolsep}{10pt}
	\setlength{\extrarowheight}{3pt}
	\renewcommand{\arraystretch}{1.2}
	
	\begin{tabularx}{\textwidth}{|c|C{0.35\textwidth}|>{\raggedright\arraybackslash}X|>{\raggedright\arraybackslash}X|}
		\hline
		№ & Входной граф (матрица смежности) & Ожидаемый вывод & Фактический вывод \\ \hline
		
		1 &
		\(
		\begin{pmatrix}
			0 & 1  & 10 \\
			10 & 0 & 1  \\
			10 & 10 & 0
		\end{pmatrix}
		\)
		&
		Маршрут: $0 \to 1 \to 2$, $L_{\text{best}} = 2$. &
		Маршрут: $0 \to 1 \to 2$, $L_{\text{best}} = 2$. \\ \hline
		
		2 &
		\(
		\begin{pmatrix}
			0 & 1 & \infty & \infty \\
			5 & 0 & 1      & \infty \\
			\infty & 5 & 0 & 1 \\
			\infty & \infty & 5 & 0
		\end{pmatrix}
		\)
		&
		Маршрут: $0 \to 1 \to 2 \to 3$, $L_{\text{best}} = 3$. &
		Маршрут: $0 \to 1 \to 2 \to 3$, $L_{\text{best}} = 3$. \\ \hline
		
		3 &
		\(
		\begin{pmatrix}
			0
		\end{pmatrix}
		\)
		&
		Маршрут: $0$, $L_{\text{best}} = 0$. &
		Маршрут: $0$, $L_{\text{best}} = 0$. \\ \hline
		
		4 &
		\(
		\begin{pmatrix}
			0 & 1 & \infty \\
			1 & 0 & \infty \\
			\infty & \infty & 0
		\end{pmatrix}
		\)
		&
		Незамкнутого маршрута не существует,
		$L_{\text{best}} = \infty$. &
		Незамкнутого маршрута не существует,
		$L_{\text{best}} = \infty$. \\ \hline
		
	\end{tabularx}
\end{table}




\section{Вывод}

В этом разделе были описаны средства реализации алгоритмов. Также были продемонстрированы листинги реализаций: полного перебора и муравьиного алгоритма. Приведены результаты тестирования.

\clearpage
