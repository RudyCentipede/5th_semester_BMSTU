\chapter{Конструкторская часть}

\section{Функциональные требования}

Ориентированный граф с элитными муравьями, карта для воздухоплавателей, время с учётом направления ветра, горы пролететь невозможно.

\textbf{Входные данные:}
\begin{itemize}
	\item граф в виде матрицы смежности;
	\item $t_{max}$ --- время жизни колонии муравьёв (количество итераций);
	\item $\alpha$ --- коэффициент влияния феромона;
	\item $\beta$ --- коэффициент влияния веса ребра;
	\item $\rho$ --- коэффициент испарения феромона;
	\item $Q$ --- квота феромона 1 муравья на 1 день;
	\item $e$ --- количество элитных муравьёв.
	
\end{itemize}

\textbf{Выходные данные:}

\begin{itemize}
	\item оптимальный маршрут;
	\item длина оптимального маршрута.
\end{itemize}

\section{Разработка алгоритмов}

В данном разделе представлены схемы алгоритмов работы: полного перебора и муравьиного алгоритма с элитными муравьями.

\subsection{Алгоритм полного перебора}

На рисунке~\ref{brute} представлена схема работы алгоритма полного перебора для решения задачи коммивояжёра.

\begin{figure}[h]
	\centering
	\includegraphics[scale=0.9]{images/brute.pdf}
	\caption{Схема работы алгоритма полного перебора}
	\label{brute}
\end{figure}

\clearpage



\subsection{Муравьиный алгоритм с элитными муравьями}

На рисунке~\ref{ant} представлена схема работы муравьиного алгоритма с элитными муравьями для решения задачи коммивояжёра.


\begin{figure}[h]
	\centering
	\includegraphics[scale=0.86]{images/ant.pdf}
	\caption{Схема работы муравьиного алгоритма с элитными муравьями}
	\label{ant}
\end{figure}

\clearpage

\section{Модель вычислений}

Чтобы вычислить трудоёмкость алгоритмов, введена следующая модель вычислений:

\begin{enumerate}
	\item базовые операции:
	\begin{itemize}
		\item трудоёмкость операций из списка~\ref{costOne} равна 1:
		\begin{equation}
			\label{costOne}
			\begin{gathered}
				=, +, \mathrel{+}=, -, \mathrel{-}=, ++, --, ==, \mathrel{!}=, <, <=, >=, >, [], \&\&,\\
				\&, >>, <<, ||, |;
			\end{gathered}
		\end{equation}
		\item трудоёмкость операций из списка~\ref{costTwo} равна 2:
		\begin{equation}
			\label{costTwo}
			\begin{gathered}
				*, \mathrel{*}=, /, \mathrel{/}=, \%, \mathrel{\%}=;
			\end{gathered}
		\end{equation}
	\end{itemize}
	\item трудоёмкость условного перехода равна 0;
	\item трудоёмкость условного оператора по формуле~(\ref{if}):
	\begin{equation}
		\label{if}
		f_{if} = f_{\text{условия}} + 
		\begin{cases}
			min(f_A, f_B) & \text{--- лучший случай,}\\
			max(f_A, f_B) & \text{--- худший случай;}
		\end{cases}
	\end{equation}
	\item трудоёмкость цикла по формуле~(\ref{for}):
	\begin{equation}
		\label{for}
		\begin{gathered}
			f_{for} = f_{\text{инициализации}} + f_{\text{сравнения}} + M\cdot(f_{\text{тела}} + \\+ f_{\text{инкремента}} + f_{\text{сравнения}}),
		\end{gathered}
	\end{equation}
	где $M$ --- число итераций.
\end{enumerate}

\section{Трудоёмкость алгоритмов}

\subsection{Трудоёмкость алгоритма полного перебора}

Трудоёмкость алгоритма полного перебора для худшего случая рассчитана по формуле~(\ref{eq:brut}):

\begin{equation}
	\label{eq:brut}
	\begin{gathered}
		f = 2 + N\cdot (2 + 2) + 1 + N!\cdot (2 + 2 + (N - 1)\cdot (3 + 3 + 4 + 2) +\\ + 2 + 2 + 2) = 3 + 4N + 12N\cdot N! - 2\cdot N!,
	\end{gathered}
\end{equation}

где $N$ --- количество вершин графа.

Асимптотическая оценка временной сложности алгоритма: $O(N\cdot N!)$.


\subsection{Трудоёмкость муравьиного алгоритма}

Пусть $T$ --- число итераций, $N$ --- количество вершин графа, $A = N$ --- количество муравьёв.

Трудоёмкость муравьиного алгоритма рассчитана по формуле~(\ref{eq:ant0}):

\begin{equation}
	\label{eq:ant0}
	\begin{gathered}
		f = f_{initVis} + f_{initPher} + 2 + T\cdot (f_{path} + f_{findBest} + f_{upd} + 2),
	\end{gathered}
\end{equation}
где $f_{initVis}$ --- трудоёмкость инициализации матрицы видимости, $f_{initPher}$ --- трудоёмкость инициализации матрицы феромона, $f_{path}$ --- трудоёмкость построения маршрутов муравьями, $f_{findBest}$ --- трудоёмкость нахождения лучшего маршрута, $f_{upd}$ --- трудоёмкость обновления матрицы феромона.

Трудоёмкость инициализации матрицы видимости рассчитана по формуле~(\ref{eq:ant1}):

\begin{equation}
	\label{eq:ant1}
	\begin{gathered}
		f_{initVis} = 2 + N\cdot (4 + 16N) = 2 + 4N + 16N^2.
	\end{gathered}
\end{equation}

Трудоёмкость инициализации матрицы феромона рассчитана по формуле~(\ref{eq:ant2}):

\begin{equation}
	\label{eq:ant2}
	\begin{gathered}
		f_{initPher} = 2 + N\cdot (4 + 12N) = 2 + 4N + 12N^2.
	\end{gathered}
\end{equation}

Трудоёмкость построения маршрутов муравьями рассчитана по формуле~(\ref{eq:ant3}):

\begin{equation}
	\label{eq:ant3}
	\begin{gathered}
		f_{path} = 2 + N\cdot (11 + N\cdot (6 + N\cdot (8 + 3 + 3 + 6 + 6))) = \\ = 2 + N\cdot (11 + 6N + 26N^2) = 2 + 11N + 6N^2 + 26N^3.
	\end{gathered}
\end{equation}

Трудоёмкость нахождения лучшего маршрута рассчитана по формуле~(\ref{eq:ant4}):

\begin{equation}
	\label{eq:ant4}
	\begin{gathered}
		f_{findBest} = 2 + 2 + N\cdot (2 + 2 + 1 + 2) + 2 + 2 + N\cdot (4 + 2) =\\ = 8 + 13N.
	\end{gathered}
\end{equation}

Трудоёмкость обновления матрицы феромона рассчитана по формуле~(\ref{eq:ant5}):

\begin{equation}
	\label{eq:ant5}
	\begin{gathered}
		f_{upd} = 2 + N\cdot (2 + N\cdot (2 + 1 + 2 + 2) + 2) + 2 + N\cdot (2 + \\ + 2 + 1 + 2 + 1 + 2 + (N - 1)\cdot (3 + 4 + 3 + 2) + 2) + 4 + 1 + \\ + 2 +(N - 1)\cdot (2 + 3 + 3 + 2) = 1 + 14N + 19N^2.
	\end{gathered}
\end{equation}

Общая трудоёмкость рассчитана по формуле~(\ref{eq:ant6}):

\begin{equation}
	\label{eq:ant6}
	\begin{gathered}
		f = 2 + 4N + 16N^2 + 2 + 4N + 12N^2 + 2 + T\cdot (2 + 11N + 6N^2 + \\ + 26N^3 +  8 + 13N + 1 + 14N + 19N^2 + 2) = 6 + 8N + 28N^2 + \\ + 13T + 38TN + 25TN^2 + 26TN^3.
	\end{gathered}
\end{equation}

Асимптотическая оценка временной сложности алгоритма: $O(T\cdot N^3)$.

\section{Вывод}

В данном разделе были описаны функциональные требования к программе, построены схемы работы алгоритмов решения задачи коммивояжёра: метода полного перебора и муравьиного алгоритма, выполнена оценка трудоёмкости алгоритмов.
\clearpage
