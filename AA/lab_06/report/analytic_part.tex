\chapter{Аналитическая часть}

\section{Ориентированный граф}

Ориентированный граф $G$ задаётся двумя множествами~(\ref{eq1})

\begin{equation}
	\label{eq1}
	G = (V, E),
\end{equation}

где $V$ --- конечное множество, элементы которого называют вершинами или узлами, $E$ --- множество упорядоченных пар на $V$, то есть подмножество множества $V\times V$, элементы которого называют дугами~\cite{book_belousov}.

Если дуга $e = (u, v)$, то говорят, что дуга $e$ ведёт из вершины $u$ в вершину $v$, и обозначают это $u\rightarrow v$~\cite{book_belousov}.

\section{Задача коммивояжёра}

Задача коммивояжёра --- одна из самых известных задач комбинаторной оптимизации, заключающаяся в поиске самого выгодного маршрута, проходящего через указанные города ровно по одному разу с последующим возвратом в исходный город.

В данной лабораторной работе задача коммивояжёра подразумевает
нахождение незамкнутого маршрута, то есть нужно посетить все города, закончить в любом другом городе.

\subsection{Метод полного перебора}

Метод полного перебора подразумевает рассмотрение всех возможных
маршрутов путём перестановок вершин. Гарантируется, что на выходе будет самый короткий маршрут.

Достоинством данного метода является гарантия получения оптимального маршрута.

Недостатком данного метода является его алгоритмическая
сложность $O(n!)$.

\subsection{Метод на основе муравьиного алгоритма}

Моделирование поведения муравьёв связано с распределением феромона на 
тропе --- ребре графа в задаче коммивояжёра. При этом вероятность включения 
ребра в маршрут отдельного муравья пропорциональна количеству феромона на 
этом ребре, а количество откладываемого феромона пропорционально длине 
маршрута. Чем короче маршрут, тем больше феромона будет отложено на его 
рёбрах, следовательно, большее количество муравьёв будет включать его в синтез 
собственных маршрутов. Моделирование такого подхода, использующего только 
положительную обратную связь, приводит к преждевременной сходимости --- 
большинство муравьёв двигается по локально оптимальному маршруту. Избежать 
этого можно, моделируя отрицательную обратную связь в виде испарения феромона. При этом если феромон испаряется быстро, то это приводит к потере памяти колонии и забыванию хороших решений, с дугой стороны, большое время испарения может привести к получению устойчивого локально оптимального решения~\cite{book_ant}.

Муравьи имеют собственную <<память>>. Поскольку каждый город может 
быть посещён только один раз, у каждого муравья сеть список уже посещённых городов --- список запретов.

Муравьи обладают <<зрением>> --- видимость есть эвристическое желание 
посетить город j, если муравей находится в городе i.

Муравьи обладают <<обонянием>> --- они могут улавливать след феромона, 
подтверждающий желание посетить город j из города i, на основании опыта 
других муравьёв.

Дополнительная модификация алгоритма может состоять во введении 
так называемых <<элитных>> муравьёв, которые усиливают ребра наилучшего маршрута, найденного с начала работы алгоритма~\cite{book_ant}.

\section{Вывод}

В аналитической части были рассмотрены понятие ориентированного графа, задача коммивояжёра и методы её решения.  

\clearpage
