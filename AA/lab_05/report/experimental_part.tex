\chapter{Исследовательская часть}

\section{Технические характеристики ЭВМ}

Замеры времени проводились на ноутбуке ACER Predator со следующими техническими характеристиками:
\begin{itemize}
	\item процессор Intel(R) Core(TM) i7-10750H с тактовой частотой 2.60ГГц;
	\item ОЗУ 16 ГБ;
	\item ОС Windows 10 Pro 64 разрядная;
	\item 12 логических ядер.
\end{itemize}
\bigskip

Во время замеров времени ноутбук был подключён к электропитанию, сторонними приложениями нагружен не был. 

\section{Замеры времени}

Были проведены замеры времени обработки набора заявок для двух реализаций: последовательной и конвейерной параллельной (три обслуживающих устройства ОУ1 --- ОУ3, работающих в отдельных потоках и обменивающихся заявками через очереди, при этом ОУ2 использует вспомогательные рабочие потоки).

Измерения проводились для 
$N \in\{25,50,75,100,125\}$ заявок при $k = 7$ потоках.

В таблице~\ref{tab:pipe_vs_seq} представлены результаты замеров времени обработки заявок последовательно и с помощью конвейера.

\begin{table}[H]
	\centering
	\caption{Результаты замеров времени обработки заявок последовательно и с помощью конвейера}
	\label{tab:pipe_vs_seq}
	\begin{tabular}{|c|c|c|}
		\hline
		\textbf{Количество заявок} & \textbf{Последовательно, мс} & \textbf{Конвейер, мс} \\
		\hline
		25  & 1387.000 & 554.004 \\
		\hline
		50  & 2820.000 & 1110.000 \\
		\hline
		75  & 4199.000 & 2155.000 \\
		\hline
		100 & 5543.000 & 2276.000 \\
		\hline
		125 & 7623.000 & 2815.000 \\
		\hline
	\end{tabular}
\end{table}

На рисунке~\ref{fig:pipe_vs_seq} изображены графики зависимостей времени последовательной и конвейерной обработки набора заявок от их количества.

\begin{figure}[H]
	\centering
	\begin{tikzpicture}
		\begin{axis}[
			xlabel={Количество заявок $N$},
			ylabel={Время, мс},
			grid=both,
			legend pos=north west,
			width=0.9\linewidth,
			height=0.5\linewidth,
			yticklabel style={xshift=-3pt},
			ylabel style={at={(axis description cs:-0.02,0.5)},anchor=south},
			]
			\addplot coordinates {(25,1387) (50,2820) (75,4199) (100,5543) (125,7623)};
			\addlegendentry{Последовательно}
			
			\addplot coordinates {(25,554.004) (50,1110) (75,2155) (100,2276) (125,2815)};
			\addlegendentry{Конвейер}
		\end{axis}
	\end{tikzpicture}
	\caption{Зависимости времени последовательной и конвейерной обработки набора заявок от их количества}
	\label{fig:pipe_vs_seq}
	
\end{figure}

В среднем конвейер обрабатывал заявки в 2,4 раза быстрее, чем при последовательной обработке.



\section{Вывод}

В данном разделе были описаны технические характеристики машины,
на которой проводились замеры времени. Продемонстрированы результаты замеров времени обработки набора заявок для двух реализаций: последовательной и конвейерной.


\clearpage
