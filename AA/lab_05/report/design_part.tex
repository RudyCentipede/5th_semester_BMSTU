\chapter{Конструкторская часть}

\section{Функциональные требования}

Выполнить кластеризацию вершин графа по числу соседей, находящихся на расстоянии 1 ребра, 2 рёбер, ..., M рёбер (M --- входной параметр). Алгоритм DBSCAN.

\textbf{Входные данные:}
\begin{itemize}
	\item файл, содержащий граф в формате graphviz;
	\item расстояние M;
	\item минимальное число точек на расстоянии M, чтобы точка считалась ядром.
\end{itemize}

\textbf{Выходные данные:}

\begin{itemize}
	\item кластеризованный граф;
	\item количество кластеров;
	\item количество вершин в каждом кластере.
\end{itemize}

\section{Разработка алгоритмов}

В данном разделе представлены схемы алгоритмов работы: последовательного и параллельного DBSCAN, рабочего потока, конвейера обработки данных.

\subsection{Последовательный алгоритм DBSCAN}

На рисунках~\ref{alg:db1} --~\ref{alg:db3} представлена схема работы последовательного алгоритма DBSCAN.

\begin{figure}[h]
	\centering
	\includegraphics[scale=0.9]{images/dbscan_1.pdf}
	\caption{Схема работы последовательного алгоритма DBSCAN (часть 1)}
	\label{alg:db1}
\end{figure}

\clearpage

\begin{figure}[h]
	\centering
	\includegraphics[scale=0.9]{images/dbscan_2.pdf}
	\caption{Схема работы последовательного алгоритма DBSCAN (часть 2)}
	\label{alg:db2}
\end{figure}

\clearpage

\begin{figure}[h]
	\centering
	\includegraphics[scale=0.9]{images/dbscan_3.pdf}
	\caption{Схема работы последовательного алгоритма DBSCAN (часть 3)}
	\label{alg:db3}
\end{figure}

\clearpage

\subsection{Параллельный алгоритм DBSCAN}

Модифицированный алгоритм делает параллельно операцию поиска расстояний до достижимых вершин (находящихся не дальше расстояния M).

На рисунках~\ref{alg:db_prl1} --~\ref{alg:db_prl3} представлена схема работы параллельного алгоритма DBSCAN.


\begin{figure}[h]
	\centering
	\includegraphics[scale=0.9]{images/dbscan_prl_1.pdf}
	\caption{Схема работы параллельного алгоритма DBSCAN (часть 1)}
	\label{alg:db_prl1}
\end{figure}

\clearpage

\begin{figure}[h]
	\centering
	\includegraphics[scale=0.9]{images/dbscan_prl_2.pdf}
	\caption{Схема работы параллельного алгоритма DBSCAN (часть 2)}
	\label{alg:db_prl2}
\end{figure}

\clearpage

\begin{figure}[h]
	\centering
	\includegraphics[scale=0.9]{images/dbscan_prl_3.pdf}
	\caption{Схема работы параллельного алгоритма DBSCAN (часть 3)}
	\label{alg:db_prl3}
\end{figure}

\clearpage

\subsection{Рабочий поток}

На рисунке~\ref{alg:worker} представлена схема алгоритма работы рабочего потока.

\begin{figure}[h]
	\centering
	\includegraphics[scale=0.85]{images/worker.pdf}
	\caption{Схема алгоритма работы рабочего потока}
	\label{alg:worker}
\end{figure}

\clearpage

\subsection{Конвейер}

На рисунке~\ref{alg:pipeline} представлена схема работы конвейера.

\begin{figure}[h]
	\centering
	\includegraphics[scale=0.9]{images/pipeline.pdf}
	\caption{Схема работы конвейера}
	\label{alg:pipeline}
\end{figure}


\section{Вывод}

В данном разделе были описаны функциональные требования к программе, построены схемы работы алгоритмов: последовательного DBSCAN, параллельного алгоритма DBSCAN, рабочего потока, конвейера обработки данных.
\clearpage
